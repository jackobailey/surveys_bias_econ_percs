\begin{table}

\caption{\label{tab:tab1}There is a strong relationship between voting behaviour at the 2016 referendum on European Union membership and the 2017 general election. Even so, this relationship is not absolute. For example, some who voted for the incumbent Conservative Party in 2017 voted to remain in the EU in 2016. Likewise, some who voted for an opposition party voted to leave the EU.}
\centering
\begin{tabular}[t]{lrrrr}
\toprule
  & \textsf{Nonvoter} & \textsf{Incumbent} & \textsf{Opposition} & \textsf{Total}\\
\midrule
\textsf{Nonvoter} & 60.7\% (495) & 2.2\%  (20) & 6.2\%  (56) & 21.6\%  (571)\\
\textsf{Leave} & 20.4\% (166) & 70.5\% (648) & 26.4\% (239) & 39.9\% (1053)\\
\textsf{Remain} & 18.9\% (154) & 27.3\% (251) & 67.4\% (611) & 38.5\% (1016)\\
\textsf{Total} & 100.0\% (815) & 100.0\% (919) & 100.0\% (906) & 100.0\% (2640)\\
\bottomrule
\end{tabular}
\end{table}